%!TEX root = ../dissertation.tex
\chapter{Conclusion}
\label{chap:conclusion}
% 1. Restate your thesis statement. Rephrase it so that slightly different from the thesis statement presented in the introduction and does not sound repetitive. 
% 2. Reiterate the key points of your work. To do this, go back to your thesis and extract the topic sentences of each main paragraph/argument. Rephrase these sentences and use them in your conclusion.
% 3. Explain the relevance and significance of your work. These should include the larger implications of your work and showcase the impact it will have on society.
% 4. End with a take-home message, such as a call to action or future direction.



%%%%%%%%%%%%%%%%%%%
% Answer main RQ
% Summarize findings and implications for others
% Summarize what others can take from the work
% Outline future work
%%%%%%%%%%%%%%%%%%%

This study aimed to identify how open source publications on GitHub can be used to infer actionable recommendations for \acrshort{rse} practice to improve the research software landscape of an organization. 
In order to do so, we reviewed the \acrshort{fair} principles, identified suitable variables to measure FAIRness, and conducted an exploratory data analysis.
The quantitative analysis was applied to \acrshort{uu} to determine different characteristics in the faculties, support for the application of \acrshort{fair} variables, and how well research software can be identified. 

Our method retrieved 176 users that had 1521 repositories. 823 of the repositories can be considered research software.
We found that proposed \acrshort{fair} variables are a helpful addition to measuring FAIRness and that there are different characteristics in the faculties, indicating a need for different possibilities of support for applying \acrshort{fair} variables. 
Among other findings, we found out that Geosciences have 57\% of unlicensed software, while the next highest percentage is much lower with 35\% for the Humanities. There is also a clear difference in language usage between Social Sciences, who primarily use R, and the other faculties, who primarily use Python.
We additionally provided first models for classifying research software to facilitate future research software identification, achieving an accuracy of 70\%.
The \acrshort{fair} data analysis for GitHub allows \acrshort{uu} to make data-based decisions, as a first analysis of the research software landscape at \acrshort{uu} and a first labelled dataset for reuse were provided.
This has not only confirmed and refuted beliefs in existing literature, but also provided novel findings and laid the foundation for repeated analyses of this kind.
Additionally, the \acrshort{swordsuu} framework was extended with additional validated  \acrshort{fair} variables. 

There are several topics for future work, which are explained in detail in \autoref{chap:discussion}. Data collection can be improved in several ways, analysis can be extended and applied to more variations of subpopulations, and classification of research software can be refined.
We conclude that the conducted analysis allows us to infer actionable recommendations for \acrshort{rse} practice and encourage others to reuse and improve the method.








% Step 1: Answer your research question
% Your conclusion should begin with the main question that your thesis or dissertation aimed to address. This is your final chance to show that you’ve done what you set out to do, so make sure to formulate a clear, concise answer.



% Step 2: Summarize and reflect on your research
% Your conclusion is an opportunity to remind your reader why you took the approach you did, what you expected to find, and how well the results matched your expectations.
% To avoid repetition, consider writing more reflectively here, rather than just writing a summary of each preceding section. Consider mentioning the effectiveness of your methodology, or perhaps any new questions or unexpected insights that arose in the process.
% You can also mention any limitations of your research, but only if you haven’t already included these in the discussion. Don’t dwell on them at length, though—focus on the positives of your work.
% Example: Summarization sentence
% While x limits the generalizability of the results, this approach provides new insight into y.
% This research clearly illustrates x, but it also raises the question of y.



% Step 3: Make future recommendations
% You may already have made a few recommendations for future research in your discussion section, but the conclusion is a good place to elaborate and look ahead, considering the implications of your findings in both theoretical and practical terms.
% Example: Recommendation sentence
% Based on these conclusions, practitioners should consider …
% To better understand the implications of these results, future studies could address …
% Further research is needed to determine the causes of/effects of/relationship between …
% When making recommendations for further research, be sure not to undermine your own work. Relatedly, while future studies might confirm, build on, or enrich your conclusions, they shouldn’t be required for your argument to feel complete. Your work should stand alone on its own merits.
% Just as you should avoid too much self-criticism, you should also avoid exaggerating the applicability of your research. If you’re making recommendations for policy, business, or other practical implementations, it’s generally best to frame them as “shoulds” rather than “musts.” All in all, the purpose of academic research is to inform, explain, and explore—not to demand.


% Step 4: Emphasize your contributions to your field
% Make sure your reader is left with a strong impression of what your research has contributed to the state of your field.

% Some strategies to achieve this include:

% Returning to your problem statement to explain how your research helps solve the problem
% Referring back to the literature review and showing how you have addressed a gap in knowledge
% Discussing how your findings confirm or challenge an existing theory or assumption
% Again, avoid simply repeating what you’ve already covered in the discussion in your conclusion. Instead, pick out the most important points and sum them up succinctly, situating your project in a broader context.