%!TEX root = ../dissertation.tex
\titlespacing{\chapter}{0pt}{-60pt}{0pt}
\chapter{Conclusion}
\label{chap:conclusion}
\enlargethispage{\baselineskip}
This study aimed to identify how open source publications on GitHub can be used to infer actionable recommendations for \acrshort{rse} practice to improve the research software landscape of an organization. 
In order to do so, we reviewed the \acrshort{fair} principles, identified suitable variables to measure FAIRness, and conducted an exploratory data analysis.
The quantitative analysis was applied to \acrshort{uu} to determine different characteristics in the faculties, support for the application of \acrshort{fair} variables, and how well research software can be identified. 

Our method retrieved 176 users that had 1521 repositories. 823 of the repositories can be considered research software.
We found that proposed \acrshort{fair} variables are a helpful addition to measuring FAIRness and that there are different characteristics in the faculties, indicating a need for different possibilities of support for applying \acrshort{fair} variables. 
Among other findings, we found out that Geosciences have 57\% of unlicensed software, while the next highest percentage is much lower with 35\% for the Humanities. There is also a clear difference in language usage between Social Sciences, who primarily use R, and the other faculties, who primarily use Python.
We additionally provided first models for classifying research software to facilitate future research software identification, achieving an accuracy of 70\%.
The \acrshort{fair} data analysis for GitHub allows \acrshort{uu} to make data-based decisions, as a first analysis of the research software landscape at \acrshort{uu} and a first labelled dataset for reuse were provided.
This has not only confirmed and refuted beliefs in existing literature, but also provided novel findings and recommendations, as well as laid the foundation for repeated analyses of this kind. The recommendations include expanding the R café, creating \acrshort{fair} reference documents, featuring and highlighting high impact and FAIR research software, and creating yearly reports.
Additionally, the \acrshort{swordsuu} framework was extended with additional validated  \acrshort{fair} variables. 
There are several topics for future work, which are explained in detail in \autoref{chap:discussion}. Data collection can be improved in several ways, analysis can be extended and applied to more variations of subpopulations, and classification of research software can be refined.
We conclude that the conducted analysis allows us to infer actionable recommendations for \acrshort{rse} practice and encourage others to reuse and improve the method.



\titlespacing{\chapter}{0pt}{-50pt}{0pt}