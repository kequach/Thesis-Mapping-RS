%!TEX root = ../dissertation.tex
% the abstract

% One or two sentences providing a **basic introduction** to the field,  comprehensible to a scientist in any discipline.
% Two to three sentences of **more detailed background**, comprehensible  to scientists in related disciplines.
% One sentence clearly stating the **general problem** being addressed by  this particular study.
% One sentence summarizing the main result (with the words "**here we show**" or their equivalent).
% Two or three sentences explaining what the **main result** reveals in direct comparison to what was thought to be the case previously, or how the main result adds to previous knowledge.
% One or two sentences to put the results into a more **general context**.
% Two or three sentences to provide a **broader perspective**, readily comprehensible to a scientist in any discipline.

Research software enables data processing and plays a vital role in academia and industry. As such, it is essential to have \acrfull{fair} research software. However, what precisely the landscape of research software looks like is unknown. Thus, we would like to understand the research software landscape better and utilize this information to infer actionable recommendations for the \acrfull{rse} practice.
This study provides insights into the research software landscape at Utrecht University through an exploratory analysis while also considering the different scientific domains. We achieve this by collecting GitHub data and analyzing repository FAIRness and characteristics through heatmaps, histograms, statistical tables, and tests. 
Our method retrieved 176 users with 1521 repositories, of which 823 are considered research software. Others can adopt the proposed method to gain insights into their specific organization, as it is designed to be reproducible and reusable.
The analysis showed significant differences between faculty characteristics and how to support the application of \acrshort{fair} variables. Among other things, our results showed that Geosciences have the highest percentage of unlicensed repositories with 57\%. Also, Social Sciences are an outlier in language usage, as they are the only faculty to primarily use R. Other faculties primarily use Python.
A first classification model is developed that achieves 70\% accuracy in identifying research software that can be used for future labelling tasks.
We conclude that our labelled GitHub dataset allows us to infer actionable recommendations on \acrshort{rse} practice.


% Among other findings, we found out that Geosciences have 57\% of unlicensed software, while the next highest percentage is much lower with 35\% for the Humanities. There is also a clear difference in language usage between Social Sciences, who primarily use R, and the other faculties, who primarily use Python.